\documentclass{article}

\usepackage{array}
\usepackage{etoolbox}
\usepackage{fancyhdr}
\usepackage{geometry}
\usepackage{graphicx}
\usepackage{soul}
\usepackage{titling}
\usepackage[english]{babel}
\usepackage[backend=biber]{biblatex}
\addbibresource{references.bib}

%%%%%%%%%%%%%%%%%%%%%%%%%%%%%%%%%%%%%%%%%%%%%%%%%%%%%%%%%%%%
% BEGIN METADATA: Edit the following as appropriate
%%%%%%%%%%%%%%%%%%%%%%%%%%%%%%%%%%%%%%%%%%%%%%%%%%%%%%%%%%%%

\title{IntelliWatt}%Illuminating Pakistan's Electricity Consumption and Billing*%  % the title of your project
\newcommand\shorttitle{\thetitle}  % if needed: a shorter title for the document header
% Team members.
\newcommand\firstname{Ali Asghar Yousuf}  % full name
\newcommand\firstid{ay06993}         % ID, e.g. xy01234
\newcommand\secondname{Muhammad Azeem Haider} % full name
\newcommand\secondid{mh0658}        % ID, e.g. xy01234
\newcommand\thirdname{Mohammad Shahid Mahmood}  % full name
\newcommand\thirdid{mm06600}         % ID, e.g. xy01234
% Uncomment the rows for the next 2 students if and as needed.
\newcommand\fourthname{Syed Ibrahim Ali Haider} % full name
\newcommand\fourthid{sh06565}        % ID, e.g. xy01234
% \newcommand\fifthname{Student 5}  % full name
% \newcommand\fifthid{id05}         % ID, e.g. xy01234

%%%%%%%%%%%%%%%%%%%%%%%%%%%%%%%%%%%%%%%%%%%%%%%%%%%%%%%%%%%%
% END METADATA: Do not edit the preamble any further.
%%%%%%%%%%%%%%%%%%%%%%%%%%%%%%%%%%%%%%%%%%%%%%%%%%%%%%%%%%%%

\pagestyle{fancy}
\lhead{Kaavish Proposal}
\chead{\shorttitle}
\rhead{Fall 2023}
\cfoot{Page \thepage}
\renewcommand{\footrulewidth}{0.4pt}

\newcommand\instruction[1]{\textit{#1}}

\begin{document}

% Cover page.
\input{cover}

%%%%%%%%%%%%%%%%%%%%%%%%%%%%%%%%%%%%%%%%%%%%%%%%%%%%%%%%%%%%
% DATA: Populate the rest of the document as instructed.
%%%%%%%%%%%%%%%%%%%%%%%%%%%%%%%%%%%%%%%%%%%%%%%%%%%%%%%%%%%%
\section{Abstract}
% \instruction{Please write a 500-600 word abstract on the project idea. It should not be very technically written but should be understandable by anyone.} \\
Electricity consumption in Pakistan presents a multifaceted challenge marked by
soaring costs per unit, unpredictable peak demand factors, frequent load
shedding, electricity theft, and the influence of extreme weather conditions.
These factors collectively result in escalating electricity bills for consumers
across the nation. A significant challenge faced by most users is their limited
awareness of their electricity consumption patterns, making it exceedingly
challenging to manage their usage effectively. Existing billing systems provide
minimal insights, hindering users from pinpointing areas where they can save
electricity and reduce costs. \newline\newline In response to these pressing
concerns, IntelliWatt aims to develop a mobile application tailored precisely
to the Pakistani context. Leveraging predictive analysis, artificial
intelligence (AI) concepts, and advanced data visualization techniques, our
application intends to empower users with comprehensive insights into their
electricity usage patterns. Beyond basic aggregate data, it will offer granular
room-wise consumption breakdowns and forecasts for expected bills. Crucially,
the application will provide actionable recommendations, enabling users to make
informed decisions and proactively reduce their electricity bills.
\newline\newline IntelliWatt is poised to be a transformative solution that
empowers users to manage their electricity consumption effectively, comprehend
their bills, and actively participate in sustainable energy practices. By
democratizing access to electricity consumption data, the application promotes
financial literacy, fosters responsible energy usage, and aligns with the
United Nations' sustainable goals related to reducing inequalities in access to
essential services. \newline\newline While existing solutions in Pakistan are
limited, primarily catering to users with solar panel installations,
IntelliWatt distinguishes itself by comprehensively addressing the needs of
typical grid-connected users. We are committed to adapting our application to
the unique dynamics of the Pakistani energy landscape, considering local
factors such as electricity pricing structures, peak demand factors,
load-shedding schedules, and the influence of extreme weather conditions.
Additionally, our focus on room-wise consumption analytics and personalized
recommendations sets us apart from generic energy management apps.
\newline\newline IntelliWatt will require a multidisciplinary approach, drawing
upon expertise in Machine Learning and Deep Learning for predictive analytics,
Computational Intelligence for optimization techniques, Data Science for
handling and analyzing consumption data, and accuracy validation, creating a
user-friendly application, and training accurate machine learning models with
limited data loom large. Nonetheless, we are resolute in our commitment to
addressing these challenges head-on and delivering a transformative solution
that will revolutionize the way Pakistan consumes and manages electricity.
\section{Problem definition}
% \instruction{Describe the problem that the project addresses.} \\
In Pakistan, electricity consumption poses significant challenges, including
high costs, erratic demand, frequent power outages, and a lack of awareness
among consumers regarding their usage patterns. Existing billing systems
provide limited insights and may lead to overbilling. These issues result in
financial burdens for households. To address this problem, we aim to develop
the "IntelliWatt" mobile application, tailored to the Pakistani context, to
empower users with accurate consumption data, insights, and cost-saving
recommendations. This app will help users optimize their electricity usage,
reduce bills, and contribute to financial stability
\section{Social relevance}
%\instruction{Describe any societal problem that the project addresses.}
Pakistan faces significant challenges in ensuring affordable and reliable
access to electricity for its citizens. The rising cost of electricity is a
financial burden for households, particularly low and middle-income
communities. There is a pressing need for tools that empower users to manage
their electricity consumption effectively and understand their bills.
\newline\newline IntelliWatt's social relevance extends beyond addressing the
pressing issues of rising electricity bills and the need for improved energy
expense management. It directly impacts society by offering detailed
consumption insights to users. By providing comprehensive breakdowns of
electricity usage, including room-wise analysis and expected bills, the
application empowers users to gain a deeper understanding of their consumption
patterns, enabling them to make informed decisions about their energy use.
\newline\newline Moreover, IntelliWatt employs cutting-edge predictive analysis
and artificial intelligence to accurately forecast users' future electricity
usage and bills. This predictive capability plays a vital role in helping users
plan their energy consumption efficiently, ultimately leading to cost savings.
\newline\newline The social impact is further amplified by the app's
user-specific insights. IntelliWatt tailors its recommendations and insights
based on individual user data, considering factors such as tariffs (commercial
or residential), the number of people in a household, peak factors, and
appliance usage. This high level of customization ensures that users receive
highly relevant and actionable information, enhancing their ability to manage
their energy expenses effectively. \newline\newline IntelliWatt goes even
deeper by calculating electricity usage for individual rooms or spaces within a
user's home. This granular breakdown allows users to identify which areas
contribute the most to their bills, facilitating targeted energy-saving
efforts. \newline\newline Additionally, the application fosters a sense of
community and competition by enabling users to compare their bills and
consumption with others in similar situations, such as the same region, the
same number of occupants, and similar appliances. This feature encourages
energy-saving practices and a collective effort to reduce electricity
consumption. \newline\newline In summary, IntelliWatt not only addresses
immediate societal concerns related to electricity bills but also actively
contributes to a more informed, efficient, and accountable energy consumption
ecosystem, ultimately benefiting individuals, households, and the larger
community.

\section{Originality/Novelty}

\subsection{Existing Solutions}

In Pakistan, existing solutions primarily comprise a handful of apps offered by
solar companies such as K-Electricity Solar. These applications are designed
primarily for users with solar panel installations and thus fall short of
providing comprehensive solutions for the broader audience of grid-connected
users. These existing apps lack the necessary depth and customization required
to address the intricate challenges associated with electricity consumption and
billing for typical grid users.

\subsection{Our Solution}

Our project, IntelliWatt, introduces an approach to tackle the issues
surrounding electricity consumption and billing in Pakistan. What sets us apart
from existing solutions are the following key differentiators:

\begin{itemize}
    \item \textbf{Tailored to the Pakistani Context:} IntelliWatt is meticulously tailored to the unique nuances of Pakistan's electricity landscape. We take into account local factors such as fluctuating electricity pricing structures, frequent peak demand periods, unpredictable load shedding, and the impact of extreme weather conditions on consumption. This local contextualization ensures that our solution is not only relevant but highly effective in addressing the specific challenges faced by Pakistani consumers.
    \item \textbf{Granular Room-wise Consumption Analysis:} Unlike generic energy management apps, IntelliWatt goes beyond the conventional by offering room-wise electricity consumption breakdowns. This granular level of insight allows users to identify precisely which areas or spaces within their homes contribute the most to their electricity bills. Armed with this knowledge, users can implement highly targeted energy-saving strategies, resulting in substantial cost reductions.  \item \textbf{Optimal Usage Recommendations:} Another groundbreaking feature of IntelliWatt is its ability to provide personalized, real-time recommendations for optimal electricity usage. These recommendations are driven by predictive analytics and CI, which take into account factors like weather forecasts. For example, during breezy weather, the app suggests optimal electricity consumption patterns, according to the various data we receive from the user and other parameters that our data has been trained on. For example on sunny days, it advises the optimal number of lights to be lit, while considering the presence of natural light, such that there is sufficient light, instead of just turning up all the lights. These suggestions translate directly into cost savings, allowing users to make informed decisions that positively impact their bills.
\end{itemize}

Through the following application we address the unique challenges faced by
typical grid-connected users in Pakistan, delivering a level of customization,
granularity, and predictive capability that in the current landscape of
electricity management applications is an advancement. IntelliWatt is a leap
forward in empowering Pakistani consumers to take control of their electricity
consumption and understanding their electricity billing.

\subsection{Literature Review}
This study by Amber et al. \cite{amber2021unlocking} delves into the realm of
household electricity consumption in Pakistan, with a specific focus on data
collected through a comprehensive survey within the residential sector. The
study's participants included employees, students, and homeowners affiliated
with Mirpur University of Science and Technology, totaling approximately 600
individuals from Mirpur city. The data collection process involved the use of a
structured questionnaire to capture information regarding household electrical
appliances and their patterns of use. The subsequent analysis unveiled
correlations between appliance ownership and electricity consumption,
ultimately leading to the identification of distinct household categories.
Notably, these findings revealed variations influenced by factors such as
average floor area and family size. \newline\newline In the context of
addressing the crucial challenge of forecasting the country's monthly
electricity consumption for effective energy planning, this paper
\cite{iftikhar2023multiple} employs a methodology that adeptly tackles the
intricacies of time series data. The dataset encompasses Pakistan's electricity
consumption from 1990 to 2020. The methodology involves a decomposition
process, segmenting the data into discernible components, including long-term
trends, seasonal patterns, and stochastic elements, through a variety of
techniques. These subseries are subsequently subjected to individual
forecasting using standard time series models. Notably, the Hybrid
Decomposition method emerges as the most accurate approach, surpassing other
techniques and benchmark methods. These findings emphasize the efficacy of the
proposed methodology, offering substantial enhancements in the accuracy of
electricity consumption forecasts \newline\newline The research
\cite{tso2007predicting} discussed here introduces three distinct techniques
for predicting electricity consumption. In addition to traditional regression
analysis, decision trees and neural networks are proposed as effective tools.
The research methodology involved the use of simulated data to train an
artificial neural network, enabling the prediction of energy consumption.
Additionally, a two-phase survey was conducted, targeting domestic households
with a consumption of 100 kWh or more. Surveyors diligently recorded details
such as the number of different appliances, their models, power ratings, and
usage patterns, which were tracked at half-hour intervals for a week. To
estimate energy consumption for various appliances, relevant computational
formulas were applied. The results underscore the significant impact of various
factors, including temperature, housing conditions (private or
government-subsidized homes), and family size, on household energy consumption.
\newline\newline Turning attention to the burgeoning demand for electricity in
Pakistan's Balochistan province, driven by population growth and the imperative
for sustainable energy sources, this paper \cite{urooj2017assessment} employs
simulation modeling, specifically utilizing the Long-range Energy Alternative
Planning System (LEAP). The aim is to scrutinize household electricity demand
spanning the period from 2005 to 2030. Data were meticulously collected from
diverse sources, including the Economic Survey of Pakistan and the Quetta
Electric Supply Company. The study unfolds three scenarios, encompassing a
Business-As-Usual (BAU) reference, wind energy integration, and solar energy
adoption. The results underscore the potential of wind and solar energy to
substantially reduce electricity demand, thereby enhancing Balochistan's
self-reliance and sustainability in energy production.

\section{CS contribution}
% \instruction{Describe the CS component of the project, e.g. the higher level CS courses that contribute to it.}    
Our project, IntelliWatt, draws upon a diverse set of computer science (CS)
courses and concepts to address the complex challenges of electricity
consumption prediction and billing estimation in Pakistan. The following CS
courses are instrumental in shaping our project's success:

\subsection{Machine Learning and Deep Learning}

Machine Learning (ML) and Deep Learning (DL) are foundational components of
IntelliWatt. ML algorithms, including deep neural networks, play a pivotal role
in predictive analysis. By analyzing historical electricity consumption data,
these algorithms enable us to forecast users' future usage and bills with
remarkable accuracy. The integration of DL techniques allows us to extract
meaningful insights from vast datasets, enhancing the precision of our
predictions.

\subsection{Computational Intelligence}

The field of Computational Intelligence augments IntelliWatt by optimizing
usage patterns for maximum efficiency. By leveraging computational intelligence
techniques, we ensure that our app suggests strategies that not only minimize
electricity consumption but also maintain user comfort and convenience. This CS
discipline guides us in striking a balance between lower unit consumption and
reduced bills while accommodating individual preferences.

\subsection{Data Science}

Data Science plays a pivotal role in handling and analyzing the extensive
consumption data we collect. Courses in Data Science provide us with the
expertise to preprocess, clean, and extract valuable insights from raw data.
Through data visualization and statistical analysis, we derive actionable
recommendations for users to optimize their electricity usage efficiently.

\subsection{Human-Computer Interaction}

Human-Computer Interaction (HCI) principles are woven into the fabric of
IntelliWatt's design philosophy. HCI courses guide us in creating an app
interface that is not only user-friendly but also accessible to diverse users.
We prioritize the seamless interaction between users and our application,
ensuring that individuals of all backgrounds and abilities can benefit from the
insights and recommendations we provide.

\subsection{Software Engineering}

The development of a user-friendly and feature-rich mobile application is at
the heart of IntelliWatt. App Development courses have equipped us with the
skills needed to design and build an intuitive interface that empowers users to
access their consumption data effortlessly. Our app's development spans various
technologies, including HTML, CSS, JavaScript, and Python for backend
development, ensuring a seamless user experience. \newline \newline In addition
to these core CS contributions, we remain open to integrating other relevant CS
courses and concepts as the project evolves. IntelliWatt is a testament to the
interdisciplinary nature of computer science, where diverse knowledge areas
converge to deliver innovative solutions that positively impact society.

\section{Scope and Deliverables}

The scope of our project, IntelliWatt, is ambitious yet achievable given the
team's size of four members and the year-long duration. Our project aims to
address the challenges of electricity consumption and billing in Pakistan by
leveraging advanced technologies, including machine learning and artificial
intelligence.

\subsection{Component 1: Building and Deploying the Deep Learning Model}

In Component 1, we focus on the development and deployment of a deep learning
model for accurate electricity consumption prediction. This component involves
the following key activities:

\begin{itemize}
    \item \textbf{Literature Review (7-10 days):} We will conduct a comprehensive literature review to understand existing solutions and context. Each team member will focus on specific aspects of the literature to ensure a thorough review.

    \item \textbf{Data Collection (2-3 weeks):} Collecting sufficient training data is crucial for training a high-accuracy deep learning model. The timeline for data collection may vary based on data source complexity. We will allocate the necessary resources and expertise for data gathering.
          \begin{itemize}
              \item \textbf{Survey Forms:}
                    An important part of the Data collection will be the Elaborate Survey that will be distributed to people who come from diverse segments of the Community. The forms will have the following questions but are not limited to them:
                    \begin{itemize}
                        \item Number of family members
                        \item Number of floors and rooms occupied
                        \item Current bills, plus 6-12 months prior bills too.
                        \item What are the major appliances that are used on a daily basis E.g. Air
                              conditioners, Refrigerator Cloth Iron, etc.
                    \end{itemize}
                    The form will also inquire about the usage trends of the appliances, e.g. ``For how many hours does the Air conditioner remain in use?''

                    Additionally researched how many units major appliances, i.e. refrigerators,
                    consume over a certain period of time. \newline Data points from this
                    Information harnessed form will be pipelined in a manner that is optimal for
                    the training of the Model. The Method of pipelining this data is still
                    undetermined.
              \item \textbf{Location for data collection :}As the Primary focus is on Karachi, the Forms will be distributed to people in Karachi only and the prediction model will also be specific to the unit rates of Karachi
              \item \textbf{Sources of Data :} The initial data collection will start from Habib University and will aim to receive data from reliable sources (our friends and family). The data collection may also take place on social media incase data from Habib University is not sufficient.
          \end{itemize}

    \item \textbf{Model Development:} Developing a robust deep learning model is a challenging task. With the time duration that we have, we will try to build, train, and fine-tune the model to the best of our abilities.

    \item \textbf{Main Features:} The deep learning model development includes the incorporation of key features that enhance its value and utility:

          \begin{itemize}
              \item \textbf{Detailed Consumption Insights:} Empowering users to understand their electricity consumption patterns, including room-wise breakdowns and expected bills.
              \item \textbf{Predictive Analysis:} Enabling accurate forecasts of users' future electricity usage and bills through predictive analysis and artificial intelligence.
              \item \textbf{User-Specific Insights:} Tailoring recommendations and insights based on user-specific data, such as tariffs, household size, peak factors, and appliance usage.
              \item \textbf{Individual Room Consumption:} Calculating electricity usage for individual rooms or spaces, allowing users to identify areas contributing the most to their bills and promoting targeted energy-saving efforts.
          \end{itemize}

    \item \textbf{Publication:} Documenting our project's methodology, findings, and results for publication is a significant goal. We will allocate time for research paper preparation, which will run parallel to other project activities.
\end{itemize}

\subsection{Component 2: Building User-Friendly Application}

Component 2 focuses on creating a user-friendly mobile application that
incorporates the deep learning model and additional features. This component
includes the following key activities:

\begin{itemize}
    \item \textbf{Application Development:} We will dedicate a significant portion of the project to creating a user-friendly mobile application. The application will serve as the primary interface for users and will seamlessly integrate the deep learning model for accurate predictions.

    \item \textbf{Additional Features:} In addition to the core application development, the project will focus on implementing several additional features that enhance the user experience and utility of the application:

          \begin{itemize}
              \item \textbf{Bill Alerts:} The app monitors real-time electricity consumption patterns and provides bill alerts when usage is projected to exceed a specific threshold just above the current slab rate. This proactive notification empowers users to take immediate action to save money.

              \item \textbf{Optimal Usage Patterns:} IntelliWatt offers personalized optimal usage patterns based on predictive analytics and weather forecasts. It recommends energy-saving practices, such as using natural warmth during cold weather and utilizing daylight during sunny days.

              \item \textbf{Room Sharing Optimization:} In shared living spaces, the app calculates individual electricity consumption for each occupant, promoting fairness and accountability among roommates.

              \item \textbf{Overbilling Check:} IntelliWatt incorporates an algorithm to detect potential overbilling. Users receive alerts if their bills significantly deviate from predicted usage patterns, allowing them to investigate and rectify billing discrepancies.

              \item \textbf{Continuous Tips and Recommendations:} The app provides users with ongoing tips and recommendations for reducing electricity consumption. These suggestions cover various aspects, from adjusting thermostat settings to adopting energy-efficient practices.
          \end{itemize}

    \item \textbf{Deployment:} Deploying the final application with all features is a complex process. We will allocate time for rigorous testing and debugging to ensure a seamless and reliable user experience.
\end{itemize}

\subsection{Deliverables}

Our project will deliver the following main components:

\begin{enumerate}
    \item \textbf{Deep Learning Model:} A highly accurate deep learning model for electricity consumption prediction.

    \item \textbf{Mobile Application:} A fully functional mobile application tailored to the Pakistani context, enabling electricity consumption prediction and billing estimation.

    \item \textbf{Research Paper:} A research paper documenting our project's methodology, findings, and results for publication in academic journals.
\end{enumerate}

In addition to these main components, our project will deliver a comprehensive
mobile application with advanced features, including bill alerts, optimal usage
patterns, room sharing optimization, overbilling checks, and continuous tips
and recommendations. These features will empower users to manage their
electricity consumption effectively and reduce their bills.

\section{Feasibility}
\subsection{Dataset Collection}

Acquiring a diverse and comprehensive dataset is a fundamental requirement for
training our deep learning model in the IntelliWatt project. Our team will be
responsible for the data collection process, with a specific focus on residents
of Karachi, Pakistan, aligning with the project's geographical scope.

Our dataset collection initiative is designed with the following objectives in
mind:

\begin{itemize}
    \item \textbf{Diversity:} Our primary goal is to assemble a dataset that embodies the wide-ranging demographic, social and economic diversity of Karachi. This diversity is crucial for training a deep learning model capable of accommodating the varied usage patterns and billing scenarios found in this region. This inclusive approach will enhance the robustness of our model and its ability to serve a broad user base

    \item \textbf{Appliance Usage and Billing Data:} The dataset will consist of appliance usage data and associated billing information. This combination is essential for training a model that can provide accurate predictions and valuable insights to our users.

    \item \textbf{Ethical Data Collection:} We are committed to upholding ethical standards in data collection. All data will be obtained with informed consent from participants, and stringent privacy and data protection measures will be implemented throughout the process.
\end{itemize}

Access to this diverse dataset will be facilitated through a combination of
methods, including surveys, data requests from utility companies, and voluntary
contributions from residents. Data quality will be a top priority, and our team
will meticulously curate, clean, and preprocess the dataset to ensure its
suitability for training the deep learning model.

\subsection{Hardware Requirements}

To carry out the training phase of our deep learning model effectively, we
require access to hardware equipped with a Graphics Processing Unit (GPU) or
access to a virtual machine with GPU capabilities. This hardware configuration
is essential for accelerating the training process, ensuring the model's
accuracy, and enhancing its overall efficiency. \newline\newline In this
context, we kindly request support and cooperation from our university,
specifically the esteemed Computer Science department, to assist us in securing
the necessary hardware resources. We are deeply appreciative of any assistance
provided, as it will significantly contribute to the feasibility and success of
our IntelliWatt project. \newline The combination of a diverse dataset and
access to GPU-enabled hardware resources will empower our team to build a
powerful and adaptable deep learning model, enabling IntelliWatt to deliver
valuable services to users for managing their electricity consumption and bills
effectively.

\section{Team Dynamics}
% \instruction{Justify the suitability of the team members to the project. For example, their relevant courses, projects, internships, or research.}

\begin{itemize}
    \item \textbf{Ali Asghar Yousuf} Ali serves as the Project Lead for IntelliWatt, bringing a wealth of knowledge and expertise in Computer Science to the team. His academic background includes advanced courses in Artificial Intelligence, Computational Intelligence, Software Engineering, Data Science, and Computer Graphics. These courses have provided him with a deep understanding of critical computer science concepts and theories, which are highly relevant to our project. During his internship at Pulflow, Ali had the opportunity to collaborate with industry experts and work effectively within a team environment. This hands-on experience not only enhanced his technical skills but also honed his ability to work seamlessly with others. In this project, Ali plays a crucial leadership role, leveraging his profound comprehension of team dynamics to optimize the entire team's performance across all tasks and deliverables required for both Kaavish I and Kaavish II.

    \item \textbf{Syed Ibrahim Ali Haider} Ibrahim contributes significantly to the IntelliWatt project, leveraging a diverse educational background that includes higher-level elective courses such as Artificial Intelligence, Computational Intelligence, Data Science, Social Network Analysis, Software Engineering, and Human-Computer Interaction. This broad spectrum of knowledge equips him with a strong foundation in various aspects of computer science. During a recent internship at Folio3, Ibrahim gained practical experience working in a collaborative team environment, actively contributing to product development. This hands-on experience has provided him with valuable insights and skills highly relevant to our project's goals. Ibrahim brings a leadership role within the project, with a primary focus on the development of the IntelliWatt application. His expertise in the AI domain and prior experience working in a team make him well-suited for this critical responsibility.

    \item \textbf{Muhammad Azeem Haider} Azeem's journey in Computer Science began during his sophomore year, and he has since developed a deep interest in data-related aspects of the discipline. His educational profile includes courses in Data Science, Social Network Analysis, Computer Vision, and Software Engineering. This well-rounded foundation enables him to approach various domains within computer science with confidence. As an undergraduate researcher in the Summer Tehqiq research program, Azeem gained valuable experience working alongside Professor Haseeb Shaikh. His research focused on the Cognitive Selection of Mental Representations, providing him with valuable research insights and mentorship. Within our IntelliWatt project, Azeem takes on a leadership role in the research domain. He will spearhead research efforts and play a prominent role in data modeling and prediction, leveraging his interdisciplinary knowledge and research experience.

    \item \textbf{Mohammad Shahid Mahmood} Shahid is a valuable member of our team, currently enrolled in courses like Social Network Analysis and Quantum Computing, and he has previously completed courses in Web and Mobile Development and Reinforcement Learning. Additionally, he has a keen interest in Augmented Reality and brings essential Product Designing Skills to the team. Shahid has also gained practical experience at Global Resource, where he worked on Web Development and dabbled in AR development. In the IntelliWatt project, Shahid leads the development and design of the end product, leveraging his skills in web and mobile development and his proficiency in product design. His diverse background and practical experience make him an essential contributor to the project's success.
\end{itemize}

\section{Tech stack}
% \instruction{Write details of the tech stack you will use for this project for e.g. if you are using MERN stack, you can write MongoDB, Express, React and NodeJS etc.}
We plan to use the following technology stack to build the different components
of our project.
\subsection{Mobile Application}
\begin{itemize}
    \item React Native
    \item Flutter
    \item Node.Js
    \item Python
\end{itemize}
\subsection{Deep/Machine Learning Model}
\begin{itemize}
    \item Python
    \item Google Colab
    \item Pandas
    \item Numpy
    \item Scikit-Learn
    \item Seaborn
\end{itemize}

% \section{References}
% \instruction{List your references.}
% \begin{itemize}
%     \item Amber, K. P., Ahmad, R., Farmanbar, M., Bashir, M. A., Mehmood, S., Khan, M. S., \& Saeed, M. U. (2021). Unlocking household electricity consumption in Pakistan. Buildings, 11(11), 566.
%     \item Lee, M. H. L., Ser, Y. C., Selvachandran, G., Thong, P. H., Cuong, L., Son, L. H., ... \& Gerogiannis, V. C. (2022). A comparative study of forecasting electricity consumption using machine learning models. Mathematics, 10(8), 1329.
%     \item Iftikhar, H., Bibi, N., Canas Rodrigues, P., \& López-Gonzales, J. L. (2023). Multiple Novel Decomposition Techniques for Time Series Forecasting: Application to Monthly Forecasting of Electricity Consumption in Pakistan. Energies, 16(6), 2579.
%     \item Predicting electricity energy consumption: A comparison of regression analysis, decision tree and neural networks
% \end{itemize}
\printbibliography

% External advisor undertaking.
\input{external}

\end{document}

%%% Local Variables:
%%% mode: latex
%%% TeX-master: t
%%% End:
